\documentclass{article}
\usepackage[left=3cm,right=3cm,top=2cm,bottom=2cm]{geometry} % page
                                                             % settings
\usepackage{amsmath} % provides many mathematical environments & tools
\usepackage[spanish]{babel}
\usepackage[doument]{ragged2e}

% Images
\usepackage{graphicx}
\usepackage{float}

% Code
\usepackage{listings}
\usepackage{xcolor}
\definecolor{gray}{rgb}{0.5,0.5,0.5}
\newcommand{\n}[1]{{\color{gray}#1}}
\lstset{numbers=left,numberstyle=\small\color{gray}}

\selectlanguage{spanish}
\usepackage[utf8]{inputenc}
\setlength{\parindent}{0mm}

\begin{document}

\title{Rotación simple a la izquierda del AVL}
\author{David Cabezas Berrido}
\date{\today}
\maketitle

\begin{justify}
  Comentario del código de la función rotacionSimpleAIzquierda,
  que toma como parámetro el nodo a rotar, el desequilibrado.
\end{justify}

\begin{verbatim}
template <class Tbase>
void AVL<Tbase>::rotacionSimpleAIzquierda(Nodo &n){
  assert(n!=nodonulo);
  enum hijo {izquierdo, derecho};
  hijo que_hijo;
  Nodo padre = arbolb.padre(n);
  
  cout << "RSI " << arbolb.etiqueta(n) << endl;
  
  ArbolBinario<Tbase> a;

  /*
  Podamos el subárbol derecho del nodo
  en el que se ha producido el desequilibrio
  y lo guardamos en a
  */
  arbolb.podar_derecha(n, a);
  
  ArbolBinario<Tbase> aux;
  
  if(padre!=nodonulo){  // Comprobamos que el padre no es nulo, es decir, que n no sea la raíz
    
    /*
    Podamos el nodo en el que se ha producido el desequilibrio
    y lo guardamos en aux. Anotamos si era el hijo derecho o el 
    hijo izquierdo de su padre
    */
    if(arbolb.izquierda(padre)==n){    
      arbolb.podar_izquierda(padre, aux);
      que_hijo = izquierdo;
    }
    else{
      arbolb.podar_derecha(padre, aux);
      que_hijo = derecho;
    }
    
    ArbolBinario<Tbase> b;
    /*
    Podamos el hijo izquierdo del subárbol derecho del
    nodo desequilibrado que habíamos podado y guardado
    en a y lo guardamos en b
    */
    a.podar_izquierda(a.raiz(), b);



    /*
    Insertamos la rama podada en b a la derecha
    del nodo n, el que estaba desequilibrado
    */
    aux.insertar_derecha(aux.raiz(), b);
    
    /*
    Insertamos el nodo n a la izquierda 
    del que era el hijo derecho
    */
    a.insertar_izquierda(a.raiz(), aux);
    
    //Reubicamos la rama rotada como hija del padre de n
    if(que_hijo == izquierdo){
      arbolb.insertar_izquierda(padre, a);
      n = arbolb.izquierda(padre);  // n sigue siendo el mismo nodo, pero sus datos han cambiado
    }
    else{
      arbolb.insertar_derecha(padre, a);
      n = arbolb.derecha(padre);
    }
  }
  else{
    
    /*
    En el caso en el que n es la raíz del árbol,
    la rotación se simplifica, ya que no tenemos
    que reenganchar la rama rotada como hija del padre.
    Directamente, esa rama es el árbol
    */

    aux = arbolb; // El nodo n es la raíz del árbol
    
    ArbolBinario<Tbase> b;

    /*
    Se guarda en b el subárbol a la derecha de a
    */
    a.podar_izquierda(a.raiz(), b);

    /*
    Se inserta dicho subárbol a la derecha del nodo n
    */
    aux.insertar_derecha(aux.raiz(), b);

    /*
    Se inserta n a la izquierda del que era su hijo derecho
    */
    a.insertar_izquierda(a.raiz(), aux);

    // Ahora el que era el hijo derecho es la raíz del árbol
    arbolb = a;
    n = arbolb.raiz();  // n sigue siendo el mismo nodo, pero sus datos han cambiado
  }
}
\end{verbatim}

\end{document}