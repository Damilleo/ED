\documentclass{article}
\usepackage[left=3cm,right=3cm,top=2cm,bottom=2cm]{geometry} % page
                                                             % settings
\usepackage{amsmath} % provides many mathematical environments & tools
\usepackage[spanish]{babel}
\usepackage[doument]{ragged2e}

% Images
\usepackage{graphicx}
\usepackage{float}

% Code
\usepackage{listings}
\usepackage{xcolor}
\definecolor{gray}{rgb}{0.5,0.5,0.5}
\newcommand{\n}[1]{{\color{gray}#1}}
\lstset{numbers=left,numberstyle=\small\color{gray}}

\selectlanguage{spanish}
\usepackage[utf8]{inputenc}
\setlength{\parindent}{0mm}

\begin{document}

\title{Reto 4: \\ Árboles}
\author{David Cabezas Berrido y Patricia Córdoba Hidalgo}
\date{\today}
\maketitle

\begin{justify}
  Procedimiento para guardar un árbol binario en disco de forma que
  se recupere la estructura jerárquica de manera unívoca usando el
  mínimo número de centinelas posibles.
\end{justify}

\begin{justify}
  La primera solución que se propuso fue colocar justo despues de la
  etiqueta de cada nodo, las etiquetas de sus hijos izquierdo y
  derecho, y colocar un centinela en caso de que el hijo no
  exista. Por tanto, después de una hoja, aparecen dos centinelas.
\end{justify}

\begin{justify}
  Esta solución siempre almacena $2n+1$ datos en memoria, siendo $n$
  el número de nodos del árbol. A continuación proponemos una solución
  para reducir el espacio necesario.
\end{justify}

\begin{justify}
  Nuestro método consiste en usar un centinela especial en caso de que
  no exista ni el hijo izquierda ni el hijo derrecha, es decir, en
  caso de que el nodo sea una hoja.
\end{justify}

\begin{justify}
  De esta forma consegumos ahorrar un centinela por cada hoja del
  árbol.  En el peor de los casos necesitaríamos $2n$ datos en memoria
  para reconstruir el árbol a partir del preorden de manera
  unívoca. Este sería el caso de un árbol con una sola hoja. El mejor
  caso sería el de un árbol estrictamente binario, es decir, un árbol
  en el que cada nodo que no es una hoja tiene tanto hijo izquierdo
  como derecho. Dicho árbol tiene el máximo número posible de hojas,
  $\frac{n+1}{2}$. Luego necesitaríamos
  $n+\frac{n+1}{2}=\frac{3n+1}{2}$ datos en memoria.
\end{justify}

\begin{justify}
  
\end{justify}

\end{document}
