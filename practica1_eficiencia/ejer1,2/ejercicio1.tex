\documentclass{article}
\usepackage[left=3cm,right=3cm,top=0cm,bottom=2cm]{geometry} % page
                                                             % settings
\usepackage{amsmath} % provides many mathematical environments & tools
\usepackage[spanish]{babel}
\usepackage[doument]{ragged2e}
\usepackage{graphicx}
\usepackage{float}

\selectlanguage{spanish}
\usepackage[utf8]{inputenc}
\setlength{\parindent}{0mm}

\begin{document}

\title{Práctica 1: Ejercicio 1}
\author{Patricia Córdoba y David Cabezas}
\date{\today}
\maketitle

\subsection*{Características del Hardware}
\begin{verbatim}

Memoria: 7.6 GiB
Procesador: Intel® Core™ i7-6700HQ CPU @ 2.60GHz x 8
Gráfica: Intel® HD Graphics 530 (Skylake GT2)
Sistema Operativo: ubuntu 17.04 64-bit
Disco: 474.8 GB
Compilador: GCC
Opciones de Compilación: g++ -o ordenacion ordenacion.cpp

\end{verbatim}

\subsection*{Eficiencia Teórica}

\begin{justify}
  Se trata de dos bucles anidados, de los cuales el segundo es no
  homogéneo, y el interior del mismo es $O(1)$. Para calcular la
  eficiencia, resolvemos la siguiente serie.
\end{justify}

\[
  \sum\limits_{i=0}^{n-2}(n-i-1) = \sum\limits_{i=0}^{n-2}n - \sum\limits_{i=0}^{n-2} 1 - \sum\limits_{i=0}^{n-2} i \\
  n(n-2) - \frac{n^2-3n+2}{2} - n + 1 = O(n^2)
\]

\subsection*{Eficiencia Empírica}

\begin{justify}
  Hemos representado los resultados empíricos (en tiempos\_ordenacion.dat) en la siguiente gráfica:
\end{justify}

\begin{figure}[H]
  \centering
  \includegraphics[width=0.6\textwidth]{ejer1.png}
\end{figure}

\end{document}
