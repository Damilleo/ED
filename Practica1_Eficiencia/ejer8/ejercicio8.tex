\documentclass{article}
\usepackage[left=3cm,right=3cm,top=0cm,bottom=2cm]{geometry} % page
                                                             % settings
\usepackage{amsmath} % provides many mathematical environments & tools
\usepackage[spanish]{babel}
\usepackage[doument]{ragged2e}
\usepackage{graphicx}
\usepackage{float}

\selectlanguage{spanish}
\usepackage[utf8]{inputenc}
\setlength{\parindent}{0mm}

\begin{document}

\title{Práctica 1: Ejercicio 8}
\author{Patricia Córdoba y David Cabezas}
\date{\today}
\maketitle

\subsection*{Características del Hardware}
\begin{verbatim}

Memoria: 7.6 GiB
Procesador: Intel® Core™ i7-6700HQ CPU @ 2.60GHz x 8
Gráfica: Intel® HD Graphics 530 (Skylake GT2)
Sistema Operativo: ubuntu 17.04 64-bit
Disco: 474.8 GB
Compilador: GCC
Opciones de Compilación: g++ -o mergesort mergesort.cpp

\end{verbatim}

\begin{justify}
  Hemos representado los resultados empíricos (en tiempos\_mergesort.dat) en la siguiente gráfica, junto con la curva de regresión $1.7278*10^{-8}n\log{n} + 0.000296896$:
\end{justify}

\begin{figure}[H]
  \centering
  \includegraphics[width=0.6\textwidth]{ejer8.png}
\end{figure}

\begin{justify}
  Hemos probado distintos valores del parámetro UMBRAL\_MS, y hemos
  obtenido los siguientes resultados:
\end{justify}

\begin{figure}[H]
  \caption{UMBRAL\_MS = 2}
  \centering
  \includegraphics[width=0.4\textwidth]{Umbral/2.png}
\end{figure}

\begin{figure}[H]
  \caption{UMBRAL\_MS = 10}
  \centering
  \includegraphics[width=0.4\textwidth]{Umbral/10.png}
\end{figure}


\begin{figure}[H]
  \caption{UMBRAL\_MS = 100}
  \centering
  \includegraphics[width=0.4\textwidth]{Umbral/100.png}
\end{figure}

\begin{figure}[H]
  \caption{UMBRAL\_MS = 500}
  \centering
  \includegraphics[width=0.4\textwidth]{Umbral/500.png}
\end{figure}

\begin{figure}[H]
  \caption{UMBRAL\_MS = 1000}
  \centering
  \includegraphics[width=0.4\textwidth]{Umbral/1000.png}
\end{figure}

\begin{justify}
  Tras estos resultados, podemos concluir que, cuanto mayor sea el
  valor de UMBRAL\_MS, mayor es el tiempo de ejecución. Sin embargo, si
  el valor de dicha constante es muy bajo, como cuando vale 2, la
  eficiencia también es disminuye, aunque no considerablemente.
  
\end{justify}

\end{document}