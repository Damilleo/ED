\documentclass{article}
\usepackage[left=3cm,right=3cm,top=0cm,bottom=2cm]{geometry} % page
                                                             % settings
\usepackage{amsmath} % provides many mathematical environments & tools
\usepackage[spanish]{babel}
\usepackage[doument]{ragged2e}
\usepackage{graphicx}
\usepackage{float}

\selectlanguage{spanish}
\usepackage[utf8]{inputenc}
\setlength{\parindent}{0mm}

\begin{document}

\title{Práctica 1: Ejercicio 4}
\author{Patricia Córdoba y David Cabezas}
\date{\today}
\maketitle

\subsection*{Características del Hardware}
\begin{verbatim}

Memoria: 7.6 GiB
Procesador: Intel® Core™ i7-6700HQ CPU @ 2.60GHz x 8
Gráfica: Intel® HD Graphics 530 (Skylake GT2)
Sistema Operativo: ubuntu 17.04 64-bit
Disco: 474.8 GB
Compilador: GCC
Opciones de Compilación: g++ -o ordenacion ordenacion.cpp
                         g++ -o ordenacion_mejor ordenacion_mejor.cpp
                         g++ -o ordenacion_peor ordenacion_peor.cpp

\end{verbatim}

\begin{justify}
  Hemos representado los resultados empíricos en el caso aleatorio, mejor y peor (en
  tiempos\_ordenacion.dat, tiempos\_ordenacion\_mejor.dat,
  tiempos\_ordenacion\_peor.dat respectivamente) en la siguiente gráfica:
\end{justify}

\begin{figure}[H]
  \centering
  \includegraphics[width=0.6\textwidth]{./../ejer4.png}
\end{figure}

\begin{justify}
  A pesar de lo esperado, el caso aleatorio consume más tiempo que el peor caso posible.  
\end{justify}

\end{document}