\documentclass{article}
\usepackage[left=3cm,right=3cm,top=2cm,bottom=2cm]{geometry} % page
                                                             % settings
\usepackage{amsmath} % provides many mathematical environments & tools
\usepackage[spanish]{babel}
\usepackage[doument]{ragged2e}
\usepackage{graphicx}
\usepackage{float}

\selectlanguage{spanish}
\usepackage[utf8]{inputenc}
\setlength{\parindent}{0mm}

\begin{document}

\title{Práctica 1: Ejercicio 5}
\author{Patricia Córdoba y David Cabezas}
\date{\today}
\maketitle

\subsection*{Características del Hardware}
\begin{verbatim}

Memoria: 7.6 GiB
Procesador: Intel® Core™ i7-6700HQ CPU @ 2.60GHz x 8
Gráfica: Intel® HD Graphics 530 (Skylake GT2)
Sistema Operativo: ubuntu 17.04 64-bit
Disco: 474.8 GB
Compilador: GCC
Opciones de Compilación: g++ -o ordenacion_mejor ordenacion_mejor.cpp
                         g++ -o ordenacion_mejor ordenacion_mejor.cpp (comentando el anterior 
                         algoritmo de burbuja y añadiendo el que dispone de condición de parada)

\end{verbatim}

\subsection*{Eficiencia Teórica}

\begin{justify}
  Hemos analizado la eficiencia del algoritmo de la burbuja optimizada y, en el mejor caso posible, tiene una eficiencia de $O(n)$, mientras que la otra implementación tiene una eficiencia de $O(n^2)$.
\end{justify}

\subsection*{Eficiencia Empírica}

\begin{justify}
  Hemos representado los resultados empíricos del mejor caso posible
  del algoritmo burbuja \\ (tiempos\_ordenacion\_mejor.dat) en la
  siguiente gráfica:
\end{justify}

\begin{figure}[H]
  \centering
  \includegraphics[width=0.6\textwidth]{ejer51.png}
\end{figure}

\begin{justify}
  \newpage
  Hemos representado los resultados empíricos del mejor caso posible
  algoritmo burbuja optimizado (tiempos\_ordenacion\_mejor\_2.dat) en la
  siguiente gráfica:
\end{justify}

\begin{figure}[H]
  \centering
  \includegraphics[width=0.6\textwidth]{ejer52.png}
\end{figure}

\end{document}
